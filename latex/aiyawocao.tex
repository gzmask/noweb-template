\title{my title}
\author{
        gzmask\\
	Department of Computer Science\\
        University of Regina\\
        Regina, Saskatchewan, S4S0A2, Canada
}
\date{\today}

\documentclass[12pt]{article}
\setlength{\parindent}{0in}
\usepackage{graphicx}
\usepackage{mathtools}
\usepackage{amsthm}
\usepackage{parskip}

\begin{document}
\maketitle

\begin{abstract}
this is the abstract
\end{abstract}

\section{pre works}
tools required:\\
  noweb - brew install noweb\\
  latex - install big mactex\\

instruction:\\
\begin{verbatim}
  notangle -Rcompile.sh aiyawocao.nw > compile.sh \\
  sh compile.sh\\
  replace \_ and other latex key ops after noweb output the thing to tex\\
  cd latex\\
  latex aiyawocao.tex\\
\end{verbatim}

\section{code}

\subsection{php file}
moodle 1.9 email mod preference page:
\begin{verbatim}
<<src/noti_pref_page.php>>=
  <?php
  <<moodle libs>>
  ?>
  <<form>>
@
\end{verbatim}

\subsection{header}
moodle 1.9 requires:
\begin{verbatim}
<<moodle libs>>=
  require_once("../../config.php");
  require_once("lib.php");
  require_once("locallib.php");
@
\end{verbatim}

\subsection{form}
ignoring uploadlib.php and filelib.php for now.\\
html form:
\begin{verbatim}
<<form>>=
  <form name="formmessages" method="POST" action="noti_handle.php">
    <input type="radio" name="notify" value="true" />Notify me new course mail by email
    <input type="radio" name="notify" value="false" />Dont't notify me new course mail by email
    <inptu type="sbumit" value="Submit" />
  </form>
@
\end{verbatim}
 
\subsection{shell script}
lastly, we will need a shell script to generated the noweb files, with -t option to supress latex output:
\begin{verbatim}
<<compile.sh>>=
  cp aiyawocao.nw latex/aiyawocao.tex
  noweb -t aiyawocao.nw
@
\end{verbatim}

\subsection{README}
\begin{verbatim}
<<README.md>>=
##
this is noweb template that i use
##
first thing to notice is that noweb generates un-escape latex sensetive chars such as '\_' etc. After generation, those chars need to be removed manually.
@
\end{verbatim}

\begin{thebibliography}{9}

\bibitem{tem}
  First. Last,
  \emph{essay title}.
  ISBN 0-000-00000-0,
  Publisher, Inc.
  Edition,
  year.

\end{thebibliography}

\end{document}
